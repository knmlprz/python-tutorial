\documentclass{article}

% Page margins
\usepackage{geometry}
\newgeometry{inner=2.54cm, outer=2.54cm, top=2.54cm,bottom=2.54cm, bindingoffset=0cm}

% Language support (date, contents, etc.)
\usepackage[polish]{babel}
\usepackage{polski}
\usepackage[utf8]{inputenc}

% Definition
\newtheorem{definition}{Definicja}

\begin{document}
	
	\section{Czym jest \LaTeX?}
	
	Jeżeli wejdziemy do wikipedii, to zobaczymy jakieś bardzo mądre zdanie: 
	
	\begin{definition}
		LaTeX – oprogramowanie do zautomatyzowanego składu tekstu, a także związany z nim język znaczników, służący do formatowania dokumentów tekstowych i tekstowo-graficznych (na przykład: broszur, artykułów, książek, plakatów, prezentacji, a nawet stron HTML).
	\end{definition}
	
	\noindent Tłumacząc to prościej, jest to inny sposób spojrzenia na przygotowanie dokumentu od znanego dla wszystkich MS Word. Jestem pewien, że nie raz męczyliście się, aby ustawić jakiś obrazek, tabelkę, nagłówek lub po prostu zwykłą listę w MS Word tak, aby to wyglądało sensownie. Wyobraźmy sobie, że w końcu udało wam się ustawić ten nagłówek tak, jak wam się podoba! Super! Ale tu nagle dopisujecie jeszcze jedno zdanie do poprzedniego rozdziału i wszystkie formatowanie wam się posypało... Znajoma sytuacja? Okazało się, że można tego uniknąć. System \LaTeX bierze na siebie odpowiedzialność za to, jak będzie wyglądał wasz dokument, a wy tylko wpisujecie treść. 
	
	
	
	\section{Ciągi znaków i  pliki tekstowe?}
	\ldots
	
\end{document}